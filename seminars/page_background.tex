\documentclass[14pt, a4paper]{extarticle}

\usepackage{sansmathfonts}
%\usepackage[T1]{fontenc}
\renewcommand*\familydefault{\sfdefault}

\usepackage{amsmath,amssymb,amsthm,minibox,graphicx,tikz,qrcode}
\usetikzlibrary{fit,shadings,calc}
\usepackage[table,dvipsnames]{xcolor}
\usepackage[protrusion=true]{microtype}

\definecolor{hkustblue}{HTML}{153870}
\definecolor{hkustyellow}{HTML}{916821}

\setlength\parindent{0pt}

\newcounter{magicrownumbers}
\newcommand\rownumber{\stepcounter{magicrownumbers}\Roman{magicrownumbers}}

\newcommand{\logo}{
  \includegraphics{HKUST_logo.pdf}
}

\usepackage[
  a4paper,
  textheight=23.7cm,
  textwidth=14.3cm,
	bottom=1.5cm
]{geometry}


\frenchspacing
\begin{document}
\pagenumbering{gobble}
% Banner background
\begin{tikzpicture}[remember picture, overlay]
  % \fill[hkustblue!25] (current page.north west) rectangle (current page.south east);
  \node[inner sep=0,fit=(current page)] (cp){};
  \shade[
    upper left=hkustblue!40,
    lower left=hkustblue!85!hkustyellow!25,
    upper right=hkustblue!93!hkustyellow!23,
    lower right=hkustblue!27
  ](cp.north west) rectangle +(4.01cm,-5cm);

  \shade[
    upper left=hkustblue!93!hkustyellow!23,
    lower left=hkustblue!27,
    upper right=hkustblue!99!hkustyellow!26,
    lower right=hkustblue!27
  ]([xshift=4cm]cp.north west) rectangle +(4.01cm,-5cm);

  \shade[
    upper left=hkustblue!99!hkustyellow!26,
    lower left=hkustblue!27,
    upper right=hkustblue!87!hkustyellow!24,
    lower right=hkustblue!30
  ]([xshift=2*4cm]cp.north west) rectangle +(4.01cm,-5cm);

  \shade[
    upper left=hkustblue!87!hkustyellow!24,
    lower left=hkustblue!30,
    upper right=hkustblue!21,
    lower right=hkustblue!32
  ]([xshift=3*4cm]cp.north west) rectangle +(4.01cm,-5cm);

  \shade[
    upper left=hkustblue!21,
    lower left=hkustblue!32,
    upper right=hkustblue!84!Maroon!22,
    lower right=hkustblue!31
  ]([xshift=4*4cm]cp.north west) rectangle ([yshift=-5cm]cp.north east);
\end{tikzpicture}%
% Spiral pattern on banner
\begin{tikzpicture}[remember picture, overlay, scale=0.45]
  \newcounter{direction}

  % Initial starting point, direction, and length
  \coordinate (start) at ([xshift=4cm,yshift=-0.5cm]current page.north east);
  \def\initLength{5.}
  \def\increaseAngle{92.}
  \def\increaseLength{0.1}

  % Initialize the direction counter to zero
  \setcounter{direction}{0}

  % Loop through and draw lines with varying angles and lengths
  \foreach \i in {1,2,...,450} {
    % Calculate current direction
    \pgfmathtruncatemacro{\tempDirection}{mod(\thedirection+\increaseAngle,360)}
    \setcounter{direction}{\tempDirection}

    % Calculate the current length for the line segment
    \pgfmathsetmacro{\currentLength}{\initLength + \increaseLength*\i}

    % Compute color variation based on loop iteration
    \pgfmathsetmacro{\colorFactor}{mod(\i,451)/450}
    \pgfmathsetmacro{\colorFactorA}{1-\colorFactor/2}
    \pgfmathsetmacro{\colorFactorB}{\colorFactor/2}
    \pgfmathsetmacro{\colorFactorC}{\colorFactor*1.2}
    \definecolor{currentColor}{rgb}{\colorFactorA, \colorFactorB, \colorFactorC}

    % Draw a line
    \draw[color=currentColor, line cap=round, line width=0.5mm, draw opacity=(1-\colorFactor)/9.5] (start) -- ++(\thedirection:\currentLength) coordinate (start);
  }
\end{tikzpicture}%
% White rectangle to cover the rest of page
\begin{tikzpicture}[remember picture, overlay]
  \fill[white] ([yshift=-\paperheight/6.1]current page.north west) rectangle (current page.south east);
\end{tikzpicture}%
% Sunflower pattern
\begin{tikzpicture}[remember picture, overlay, scale=0.32]
  \node[inner sep=0,fit=(current page)] (cp){};
  \def\nbrcircles {350}
  \def\outerradius {30mm}
  \def\deviation {.9}
  \def\fudge {.62}

  \newcounter{cumulArea}
  \setcounter{cumulArea}{0}
  \pgfmathsetmacro {\goldenRatio} {(1+sqrt(5))}
  \pgfmathsetmacro {\meanArea} {pow(\outerradius * 10 / \nbrcircles, 2) * pi}
  \pgfmathsetmacro {\minArea} {\meanArea * (1 - \deviation)}
  \pgfmathsetmacro {\midArea} {\meanArea * (1 + \deviation) - \minArea}

  \foreach \b in {0,...,\nbrcircles}{
    % mod() must be used in order to calculate the right angle.
    % otherwise, when \b is greater than 28 the angle is greater
    % than 16384 and an error is raised ('Dimension too large').
    % -- thx Tonio for this one.
    \pgfmathsetmacro{\angle}{mod(\goldenRatio * \b, 2) * 180}

    \pgfmathsetmacro{\sratio}{\b / \nbrcircles}
    \pgfmathsetmacro{\smArea}{\minArea + \sratio * \midArea}
    \pgfmathsetmacro{\smRadius}{sqrt(\smArea / pi) / 2 * \fudge}
    \addtocounter{cumulArea}{\smArea};

    \pgfmathparse{sqrt(\value{cumulArea} / pi) / 2}
    \filldraw[color=hkustblue!78, fill opacity=(1-\sratio)/33, draw opacity=(1-\sratio)/8] ([xshift=-5cm,yshift=-73cm] \angle:\pgfmathresult) circle [radius=\smRadius];
  }
\end{tikzpicture}%
% \begin{minipage}{0.07\textwidth}
%   \vspace{-10.2em}\hspace{-3.8em}
%   \scalebox{0.11}{\logo}
% \end{minipage}\hfill
% \begin{minipage}{0.93\textwidth}
%   \vspace{-10.2em}
%   \vspace{1.2em}
%   \scalebox{1.3}{\uppercase{\textbf{\textsf{Algebra and Geometry Seminar}}}}

%   \scalebox{1.02}{\textsf{The Hong Kong University of Science and Technology}}

%   \scalebox{1.02}{\textsf{Department of Mathematics}}
% \end{minipage}
\end{document}